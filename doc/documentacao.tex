\documentclass[a4paper, 12pt]{report}

\usepackage[utf8]{inputenc}
\usepackage[brazil]{babel}
\usepackage{hyperref}
\usepackage{graphicx}
\usepackage{tabularx}

\newcommand{\alforria}{\texttt{ALFORRIA}}
\newcommand{\ttt}[1]{\texttt{#1}}

\title{\alforria\ -- distribuição de encargos didáticos via
  modelos de programação linear inteira}

\author{Francisco Sobral, Depto. de Matemática, UEM\footnote{Autor
    correspondente, \url{fncsobral@uem.br}} \and João
  Costalonga, UFES}

\date{Última atualização -- \today}

\begin{document}

\maketitle

\chapter{Antes de começar}

\section{Estrutura}

\section{Arquivos de configuração}

\alforria\ necessita de alguns arquivos de dados para funcionar. Os
caminhos para cada arquivo são configurados no arquivo
\texttt{config/paths.cfg} e são divididos em entrada e saída.

Entrada:
\begin{itemize}
\item \ttt{GRUPOSPATH} -- Arquivo contendo os grupos de disciplinhas;
\item \ttt{PREFPATH} -- Arquivo CSV contendo as preferências de cada
  professor;
\item \ttt{SARPATH} -- Arquivo de texto contendo os dados de cada
  disciplina (horário, turmas, carga horária, etc.);
\item \ttt{ATRIBPATH} -- Arquivo CSV com as disciplinas pré-atribuídas
  a cada professor;
\item \ttt{FANTPATH} -- Arquivo CSV com as disciplinas que existem,
  mas são ministradas em conjunto com outras (sobreposição total de
  dias e horários);
\end{itemize}

Saída:
\begin{itemize}
\item \ttt{DATPATH} -- Arquivo .DAT com todos os dados analizados, que
  será utilizado pelo algoritmo de otimização;
\item \ttt{DAT2PATH} -- Arquivo (temporário) que é gerado pelo
  programa, para que o processo de otimização possa ocorrer
  iterativamente;
\item \ttt{SOLPATH} -- Arquivo de solução que será gerado pelo
  algoritmo de otimização
\end{itemize}
Um usuário não necessita modificar os parâmetros de saída, apenas os
de entrada.

\subsection{Grupos} \label{conf:grupos}

Neste arquivo estão descritos os grupos de disciplinas. Ele é usado
para calcular o grau de satisfação de um professor por ministrar uma
disciplina. Existem os grupos \textbf{canônicos} (C) e os grupos
\textbf{não canônicos} (N). A diferença entre eles é que os
professores podem dar diferentes graus de preferência aos grupos
canônicos.

Este arquivo dever estar no formato de texto simples \texttt{.txt},
com codificação UTF-8 e pode conter linhas em branco. É importante que
a ordem dos grupos seja a mesma usada no arquivo de preferências.

Cada linha do arquivo é dada por três partes
\begin{center} \tt
  C\_ou\_N Nome\_do\_grupo Lista\_de\_turmas
\end{center}

Como um exemplo, temos o grupo \textbf{canônico} CALC1 e o \textbf{não
  canônico} Algebra\_de\_Lie:
\begin{flushleft} \tt
C CALC1 199\_S1 1422 1583 2526 2870 3059 3203 4545 5172 6073\\
N Algebra\_de\_Lie	3314
\end{flushleft}
O mais importante é que as disciplinas dos grupos canônicos estejam
corretamente classificadas. Todas as disciplinas que estiverem no
arquivo de disciplinas e não forem canônicas serão automaticamente
classificadas com \texttt{N} pelo algoritmo e terão um grupo com nome
igual ao seu.

Um outro comentário importante é o fato de disciplinas anuais serem
separadas por semestre. Podemos ver que a disciplina 199 é anual, mas
o conteúdo associado ao grupo CALC1 é ministrado apenas no primeiro
semestre, por isso é necessário o uso de \texttt{\_S1} ou
\texttt{\_S2}. Geralmente o usuário do programa não precisa se
preocupar com isso, pois tais informações são extraídas
automaticamente do arquivo de classes usado pelo departamento. Para
tal tarefa, existe uma planilha do Google com as classes das
disciplinas, para as disciplinas canônicas. Essa planilha possui um
\textit{script} embutido chamado ``Gerador'' que gera no mesmo
diretório um arquivo de grupos com os dados presentes na
planilha. Rode o \textit{script} sempre ao adicionar/remover/atualizar
disciplinas, lembrando sempre que a ordem de grupos canônicos importa.

\paragraph{O que é importante saber}

\begin{itemize}
\item A ordem dos grupos canônicos no arquivo deve ser igual à ordem
  das colunas do arquivo de preferências (ver~\ref{conf:pref}).

\item Disciplinas anuais devem ser classificadas por semestre, usando
  a extensão \texttt{\_S1} e \texttt{\_S2}.

\item Antes da primeira rodada, verifique o arquivo de classes e use o
  \textit{script} para gerar um arquivo de grupos atualizado.
\end{itemize}

\subsection{Preferências} \label{conf:pref}

Cada linha do arquivo de preferências contém todas as preferências de
um professor. O arquivo é do tipo CSV, mas cada coluna é separada por
uma tabulação ao invés de espaços em branco. Geralmente, o arquivo
\texttt{.csv} é gerado automaticamente a partir de uma planilha que
foi preenchida via formulário ou manualmente. Por conta disso,
supõe-se que ela está ordenada de forma \textbf{crescente nas datas de
  preenchimento}. Caso um professor preencha 2 vezes a mesma tabela,
apenas a última entrada será considerada.

No nosso caso, a planilha é gerada pelo preenchimento de um formulário
\textit{online} do Google. Depois, é necessário fazer 3 tratamentos
básicos para que possa ser utilizada pelo programa:
\begin{enumerate}
\item Remoção dos nomes das colunas. Como foi dito, cada linha é
  referente a um professor;
\item Remoção das quebras de linha na coluna de comentários
  adicionais. A última coluna é reservada para o professor colocar
  comentários adicionais para à Comissão de Horário. Infelizmente,
  caso haja uma quebra de linha, esta é encarada como uma nova entrada
  de professor. Desta forma, deve-se colocar os comentários em uma só
  linha.
\item Verificar se não há nenhum texto estranho ao \LaTeX\ nas entradas
  fornecidas pelo professor. Por exemplo, se ele escreveu \verb+R^n+
  no texto, o \LaTeX\ vai dar erro de modo matemático na hora de
  compilar esse texto. Da mesma forma, isso frequentemente acontece
  com o \textit{underline}. No caso de endereços de e-mail, tal
  problema é tratado automaticamente no programa.
\item Salvar a planilha como um arquivo CSV, mas com a diferença que
  cada coluna será separada por uma tabulação (\texttt{TAB}) ao invés
  de vírgula. Qualquer outra opção de conversão deve ser removida
  (como separar por vírgula, colocar campos entre aspas, etc.).
\end{enumerate}

Atualmente, utilizando os serviços do Google, a forma mais fácil de
gerar tal planilha é através dos seguintes passos:

\begin{enumerate}
\item Crie um formulário com base no formulário padrão. Lembre-se que
  cada alteração tem que ser pensada, pois a ordem das perguntas
  altera a forma de leitura pelo programa;

\item Converta as respostas em uma planilha;

\item Inclua o \textit{script} de conversão na planilha e rode-o. Uma
  nova aba será criada;

\item Abra a aba e exporte-a como um arquivo \verb+.tsv+, salvando no
  diretório de preferência.
\end{enumerate}

Caso deseje-se criar o arquivo \texttt{.csv} na mão, é necessário
utilizar as informações da Tabela~\ref{conf:pref:tab}. A seguir
algumas explicações adicionais sobre algumas colunas do arquivo.

\begin{table}[ht!]
  \centering
  \begin{tabularx}{0.95\textwidth}{c|X}
    \hline
    Coluna & Explicação \\ \hline \hline

    1 & \textbf{Ignorada}. Data e hora de preenchimento da linha,
    gerada pelo Google. É usada apenas para ordenação das linhas em
    forma crescente no tempo.\\  \hline

    2 & Nome e sobrenome do professor \\ \hline

    3 & Identificação única. Geralmente o funcional do professor. \\ \hline

    4 & E-mail do professor \\ \hline

    5 & Telefone para contato \\ \hline

    6 & Quantidade de horas pré-atribuídas ao professor no primeiro
    semestre. Nessa coluna incluem-se todos os descontos por cargos,
    disciplinas (válidas) da pós-graduação. \textbf{Disciplinas
      pré-atribuídas da graduação e projetos de pesquisa não
      entram}. \\ \hline

    7 & Idem ao anterior, mas para o segundo semestre. \\ \hline

    8 & Explicação para a comissão sobre as cargas horárias
    pré-atribuídas \\ \hline

    9 & Tipo de licença a ser tirada no ano. Licença anual
    automaticamente remove o professor da distribuição. \\ \hline

    10 & Regime de trabalho: efetivo ou temporário. Um professor
    efetivo sem TIDE deve entrar como temporário. \\ \hline

    11-17 & Números inteiros de 0 a 10 representando os parâmetros de
    satisfação do algoritmo. Mais detalhes em~\ref{conf:pref:param} \\
    \hline

    18 & Grupos canônicos para os quais o professor é inapto a
    lecionar. Nesta coluna os nomes dos grupos devem ser separados por
    vírgulas. Maiúsculas, minúsculas e acentos não importam. Mais
    detalhes em~\ref{conf:pref:inap} \\ \hline

    19-31 & Números inteiros de 0 a 10 representando as preferências
    para lecionar disciplinas dos grupos canônicos. As colunas devem
    seguir a ordem das linhas do arquivo
    \verb+grupos.txt+. Ver~\ref{conf:grupos} e verificar o formulário
    \textit{online}. \\ \hline

    32 & \texttt{SIM} ou \texttt{NAO}, representando a opção por
    participar ou não das reuniões de departamento.\\ \hline

    33 & \texttt{COMPACTO} ou \texttt{ESPARSOS}, representando a opção
    por aulas compactadas (em sequência) ou com muitas janelas
    (esparso). \\ \hline

    & representando a opção do professor por detalhar ou resumir suas
    preferências pelo horário. \\ \hline
  \end{tabularx}
  \caption{\label{conf:pref:tab} Explicação de cada uma das colunas
    do arquivo de preferência dos professores.}
\end{table}

\subsubsection{Inaptidões}
\label{conf:pref:inap}

Atualmente, as opções de inaptidão que o formulário oferece são:
\texttt{CALC2A}, \texttt{CALC2B}, \texttt{EDO}, \texttt{EDP},
\texttt{NUMERICO}, \texttt{FINANCEIRA}, \texttt{MD1}, \texttt{MD2} e
\texttt{ENSINO}. Por razões óbvias, não é oferecida a possibilidade de
ser inapto a lecionar Cálculo I, Geometria Analítica, Álgebra Linear e
Matemática Básica.

\subsubsection{Parâmetros relativos de preferência}
\label{conf:pref:param}

Atualmente os professores podem escolher a importância
\underline{relativa} entre 7 parâmetros, que nortearão o algoritmo na
busca por uma solução que minimize a insatisfação geral. Cada coluna
segue a ordem a seguir, onde 0 é nenhuma importância, 5 significa
indiferença e 10 significa muita importância. As notas são sempre
normalizadas.

\begin{itemize}
\item \textbf{Disciplinas}: importância dada em lecionar as
  disciplinas canônicas que gosta;
\item \textbf{Número de disciplinas}: importância dada em não lecionar
  uma grande quantidade de disciplinas;
\item \textbf{Número de disciplinas distintas}: importância dada em
  não lecionar uma grande quantidade de grupos distintos de
  disciplinas
\item \textbf{Horário}: importância dada à satisfação das preferências
  de horário;
\item \textbf{Carga horária}: importância dada em lecionar uma carga
  horária reduzida;
\item \textbf{Janelas / Horários compactos}: importância dada em
  lecionar disciplinas em horários seguidos (compactos) ou espalhados
  (janelas); preferência por janelas ou horários compactos,
  respectivamente
\item \textbf{Manhã e noite}: importância dada à não atribuição de
  aulas de manhã e de noite em um mesmo dia.
\end{itemize}

Algumas preferências estão associadas com outras preferências, como os
horários. Ou seja, não faz sentido alguém preencher detalhadamente as
preferências por horário, por exemplo, se der importância 0 para esse
parâmetro. O sistema irá ignorar tais preferências. O mesmo pode ser
dito para Janelas e Disciplinas.

É claro que existe grande tentação em colocar 10 para tudo, mas,
devido à normalização, isso será equivalente a colocar 5 para tudo. O
preenchimento deve sempre ser feito sob o raciocínio da importância de
um parâmetro com relação os outros.

\end{document}
